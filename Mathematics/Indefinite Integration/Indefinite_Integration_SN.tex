\documentclass{article}
\usepackage{graphicx}
\usepackage{amsmath}
\usepackage{amssymb}
\usepackage[italicdiff]{physics}
\usepackage{enumerate}
\usepackage{microtype}
\DisableLigatures{encoding= *, family=*}
\usepackage{titlesec}
\usepackage{xfrac}
\setcounter{secnumdepth}{4}
\usepackage{xcolor}
\usepackage[bookmarks=false]{hyperref}
\usepackage{mathtools}
\usepackage{multicol}
\newcommand{\nsum}[1][1.4]{% only for \displaystyle
    \mathop{%
        \raisebox
            {-#1\depthofsumsign+1\depthofsumsign}
            {\scalebox
                {#1}
                {$\displaystyle\sum$}%
            }
    }
}
\newlength{\depthofsumsign}
\setlength{\depthofsumsign}{\depthof{$\sum$}}
\newlength{\totalheightofsumsign}
\newlength{\heightanddepthofargument}
\hypersetup{
    colorlinks=true,
    linkcolor=[RGB]{59 108 209},
    urlcolor=[RGB]{59 108 209}
}
\urlstyle{same}

\titleformat{\paragraph}
{\normalfont\normalsize\bfseries}{\theparagraph}{1em}{}
\titlespacing*{\paragraph}
{0pt}{3.25ex plus 1ex minus .2ex}{1.5ex plus .2ex}
\title{Indefinite Integration}
\author{}
\date{}

\begin{document}
\maketitle

\section{Fundamental Definition of \\ Indefinite Integration}
If $f$ and $F$ and functions such that $\displaystyle\dv{x}(F(x))=f(x)$ then $F$ is anti-derivative of $f$ w.r.t. $x$ symbolically, $$\displaystyle\int f(x) \,dx=F(x)+C$$
where $C$ is the constant of Integration

\section{Anti-Derivatives of Some Standard Functions}
\begin{enumerate}[i.]
    \item $\displaystyle\int k\cdot f(x) \, dx = k \cdot \displaystyle\int f(x) \, dx$
    \item $\displaystyle\int [f_{1}(x)\pm f_{2}(x)\pm f_{3}(x)\pm \ldots \pm f_{n}(x)] \, dx= \displaystyle\int f_{1}(x) \, dx\pm \displaystyle\int f_{2}(x) \, dx\pm \hphantom{1cm}\hspace{3cm} \displaystyle\int f_{3}(x) \, dx \pm \ldots \displaystyle\int f_{n}(x) \, dx$
    \item $\displaystyle\int x^n \, dx=\dfrac{x^{n+1}}{n+1} +C \hfill \hspace{2mm} n \not=-1$
    \item $\displaystyle\int \dfrac{1}{x} \, dx=\ln \abs{x} +C$
    \item $\displaystyle\int e^x \, dx=e^x +C$
    \item $\displaystyle\int a^x \, dx=\dfrac{a^x}{\ln a} +C$
    \item $\displaystyle\int \sin x \, dx=-\cos x +C$
    \item $\displaystyle\int \cos x \, dx=\sin x +C$
    \item $\displaystyle\int \sec^2 x \, dx=\tan x +C$
    \item $\displaystyle\int \csc^2 x \, dx=-\cot x +C$
    \item $\displaystyle\int \sec x \tan x \, dx=\sec x +C$
    \item $\displaystyle\int \csc x \cot x \, dx=-\csc x +C $
    \item $\displaystyle\int \cot x \, dx=\ln |\sin x| +C$
    \item $\displaystyle\int \tan x \, dx=-\ln |\cos x| +C$
    \item $\displaystyle\int \sec x \, dx=\ln |\sec x + \tan x| +C$
    \item $\displaystyle\int \csc x \, dx=\ln |\csc x-\cot x| +C$
    \item $\displaystyle\int \dfrac{1}{\displaystyle\sqrt{1-x^2}} \, dx=\sin^{-1} x +C$
    \item $\displaystyle\int \dfrac{-1}{\displaystyle\sqrt{1-x^2}} \, dx=\cos^{-1} x +C$
    \item $\displaystyle\int \dfrac{1}{1+x^2} \, dx= \tan^{-1} x+C$
    \item $\displaystyle\int \dfrac{-1}{1+x^2} \, dx= \cot^{-1}x +C$
    \item $\displaystyle\int \frac{1}{x\displaystyle\sqrt{x^2-1}} \, dx=\sec^{-1}x +C$
    \item $\displaystyle\int \dfrac{-1}{x\displaystyle\sqrt{x^2-1}} \, dx=\csc^{-1}x+C$
    \item $\displaystyle\int \sqrt{x} \, dx=\dfrac{2x\sqrt{x}}{3}+C$
    \item $\displaystyle\int \dfrac{dx}{x^2-1}=\ln \abs{\dfrac{x-1}{x+1}}+C $
    \item $\displaystyle\int \dfrac{dx}{\sqrt{1+x^2}}=\ln \abs{x+\sqrt{x^2+1}}+C$
    \item $\displaystyle\int \dfrac{dx}{\sqrt{x^2-1}}=\ln \abs{x+\sqrt{x^2-1}}+C$
    \item $\displaystyle\int \sqrt{1-x^2} \, dx=\dfrac{1}{2}x\sqrt{1-x^2}+\dfrac{1}{2} \sin^{-1} x +C $
    \item $\displaystyle\int \sqrt{x^2-1} \, dx= \dfrac{1}{2}x\sqrt{x^2-1}-\dfrac{1}{2}\ln \abs{x+\sqrt{x^2-1}}+C $
    \item $\displaystyle\int \sqrt{x^2+1} \, dx=\dfrac{1}{2}x\sqrt{x^2+1}+\dfrac{1}{2}\ln \abs{x+\sqrt{x^2+1}}+C$
    \item $\displaystyle\int \dfrac{x^2+1}{x^4+1} \, dx=\dfrac{1}{\sqrt{2}}\tan^{-1}\left(\dfrac{x-\sfrac{1}{x}}{\sqrt{2}}\right)+C$
    \item $\displaystyle\int \dfrac{x^2-1}{x^4+1} \, dx=\dfrac{1}{2\sqrt{2}}\ln \abs{\dfrac{x+\sfrac{1}{x}- \sqrt{2}}{x+\sfrac{1}{x}+\sqrt{2}}}+C$

\end{enumerate}
\textbf{Important Results}
\begin{enumerate}
    \item If $F_{1}(x)$ and $F_{2}(x)$ are two anti-derivatives of a function $f(x)$, \\then $F_{1}(x) $ and $F_{2}(x) $ only differ by a constant, i.e.

          $$F_{1}(x) - F_{2}(x)=C$$
          where, $C$ is a $\mathbb{R}$ constant.
    \item If $f(x)$ is continuous $\forall \hspace{1mm} x \in D_{f}$ and,\\
          $\displaystyle\int f(x) \, dx=F(x)+C$, then $F(x)$ always exists and is continuous.
    \item If $f(x)$ is discontinuous at $x=x_{1}$, then its anti-derivative can be continuous at $x=x_{1}$.
    \item Anti-derivative of a periodic function may not be periodic.
\end{enumerate}

\section{Methods of Integration}
\subsection{$u$ substitution}
Integrals of form, $$I=\int f(g(x))\cdot g'(x) \, dx$$

Can be solved by, the substitution,
$$u=g(x)$$ Differentiating both sides w.r.t. $x$,
$$du=g'(x)dx$$
Now, $$I=\int f(u) \, du$$
\subsection{Integrals of form $$\int \dfrac{dx}{ax^2+bx+c}, \int \dfrac{dx}{\sqrt{ax^2+bx+c}}, \int \sqrt{ax^2+bx+c} \, dx$$}
Using completing the square, $$ax^2+bx+c=a \left[\left(x+\dfrac{b}{2a}\right)+\dfrac{c}{a}-\dfrac{b^2}{4a}\right]$$
Now, using $\textit{u sub}$, let $$u=x+\dfrac{b}{2a}$$
The transformed integral can be integrated using previous methods.

\subsection{Integrals of form}
$$\int \dfrac{px+q}{ax^2+bx+c} \, dx, \int \dfrac{px+q}{\sqrt{ax^2+bx+c}} \, dx, \int \left(px+q\right)\sqrt{ax^2+bx+c} \, dx $$
$$px+q=\lambda \dv{x}(ax^2+bx+c) +\mu$$
Now, after finding $\lambda,  \mu$, \\For the 1st part, use $\textit{u sub}$,\\ Let $$u= ax^2+bx+c$$
2nd part of the integral can be integrated using previous methods.
\subsection{Integrals of form $$\int \dfrac{K(x)}{ax^2+bx+c} \, dx$$ where $deg(K(x))\ge2$}

By polynomial long division $$\dfrac{K(x)}{ax^2+bx+c} \,=Q(x) + \dfrac{R(x)}{ax^2+bx+c}$$
$deg(R(x))\le 1$,
\\ Now, the integral $$\int \dfrac{R(x)}{ax^2+bx+c} \, dx$$
can be integrated using previous methods.

\subsection{Integrals of form}
$$\int \dfrac{ax^2+bx+c}{px^2+qx+r} \, dx, \int \dfrac{ax^2+bx+c}{\sqrt{px^2+qx+r}} \, dx, \int \left(ax^2+bx+c\right)\sqrt{px^2+qx+r} \, dx$$
$$ax^2+bx+c= \lambda \left(px^2+qx+r\right)+\mu \dv{x}(px^2+qx+r)+\gamma$$
For $1$st part use Integration by Parts, 2nd and 3rd part can be integrated using previous methods.

\subsection{Trig. Integrals}
\subsubsection{Integrals of form}
$$\int \dfrac{dx}{a \cos^2 x+b\sin ^2 x}, \int \dfrac{dx}{a+b \sin^2 x}, \int \dfrac{dx}{a+b\cos^2 x}$$
$$\int \dfrac{dx}{\left(a\sin x+b\cos x\right)^2}, \int \dfrac{dx}{a+b\sin^2 x+c \cos ^2 x}$$
Steps -
\begin{enumerate}[1.]
    \item Multiply numerator and denominator by $\sec^2 x$
    \item Replace $\sec^2 x$ (if any) by $1+\tan^2 x$  except the one multiplied in step 1.
    \item Let $u=\tan x$, then $du=\sec^2 x \, dx$
\end{enumerate}
Now, the transformed integral can be integrated using previous methods.

\subsubsection{Integrals of form }
$$\int \dfrac{dx}{a\sin x +b \cos x}, \int \dfrac{dx}{a+b\sin x}, \int \dfrac{dx}{a+b\cos x}, \int \dfrac{dx}{a\sin x + b \cos x + c} $$
Steps -
\begin{enumerate}[1.]
    \item Replace $\sin x = \dfrac{2\tan \sfrac{x}{2}}{1+\tan^2 \sfrac{x}{2}}$ and $\cos x = \dfrac{1-\tan^2 \sfrac{x}{2}}{1+\tan^2 \sfrac{x}{2}}$
    \item Let $u=\tan \sfrac{x}{2}$, $du= \dfrac{1}{2} \sec^2 \sfrac{x}{2} \, dx$ is already present in the numerator.
\end{enumerate}
Now, the transformed integral can be integrated using previous methods.

\paragraph{Alternative Method to Integrate $$I=\int \dfrac{dx}{a\sin x + b \cos x}$$}

$$a\sin x + b \cos x = \sqrt{a^2+b^2} \sin (x+\tan^{-1} \left(\dfrac{b}{a}\right)) $$
$$I=\dfrac{1}{\sqrt{a^2+b^2}}\int \csc \left(x+ \tan^{-1}\left(\dfrac{b}{a}\right) \right) \, dx $$
$$I=\dfrac{1}{\sqrt{a^2+b^2}}\ln \abs{\tan \left(\dfrac{x}{2}+ \dfrac{1}{2} \tan^{-1} \left(\dfrac{b}{a}\right) \right)}+C$$
\subsubsection{Integrals of form }
$$\int \dfrac{p\cos x + q \sin x +r}{a \cos x + b \sin x + c} \, dx , \int \dfrac{p \cos x + q \sin x }{a \cos x + b \sin x} \, dx $$

Steps for (i),
\begin{enumerate}[1.]
    \item Express $\textit{Numerator}= \lambda \textit{ Denominator } + \mu \textit{ Derivative of denominator } + \gamma$
\end{enumerate}
Now, the transformed integral can be integrated using previous methods.


Steps for (ii),
\begin{enumerate}[1.]
    \item Express $\textit{Numerator}=\lambda \textit{ Denominator } + \mu \textit{ Derivative of Denominator}$
\end{enumerate}
Now, the transformed integral can be integrated using previous methods.
\subsection{Integration by parts}

$$\int uv \, dx = u \int v \, dx - \int \left(u' \int v \, dx\right) \, dx$$


$u$ is the function which has to be differentiated ($D$), $v$ is the function which has to be integrated ($I$)

\subsection{Integrals of form }
$$I=\int e^{g(x)} \left(f(x)g'(x)+f'(x)\right) \, dx$$
$I=e^{g(x)} \cdot f(x) +C$

\subsection{Integrals of form }
$$S=\int e^{ax} \sin bx \, dx, C = \int e^{ax} \cos bx \, dx$$

$S=\dfrac{e^{ax}}{a^2+b^2}\left(a \sin bx -b \cos bx\right)+C_{0}, C=\dfrac{e^{ax}}{a^2+b^2}\left(a \cos bx + b \sin bx\right) + C_{00}$

\subsection{Integration by Partial Fraction Decomposition}
Let $f(x)=\displaystyle\sum_{i=0}^{n}a_{i}x^i, g(x)=\displaystyle\sum_{i=0}^{m} b_{i}x^i$.

We define a rational function $h(x)=\dfrac{f(x)}{g(x)}$, \\ $h(x)$ is $\begin{cases}
        \textit{Proper Rational Function}   & m>n     \\
        \textit{Improper Rational Function} & m \le n
    \end{cases}$

If $h(x)$ is Improper, we make it Proper by polynomial long division, i.e. $$h(x)=Q(x)+ \dfrac{r(x)}{g(x)}$$

Clearly, $\dfrac{r(x)}{g(x)}$ is Proper.


Now, assuming $h(x)$ is Proper,
\subsubsection{$g(x)$ is the product of non-repeating linear factors}
Let $$g(x)=L_{1}(x) \cdot L_{2}(x) \cdot  \ldots \cdot L_{m}(x)$$ where $L_{i}(x)$ are linear functions.\\
Then, we can expand $\dfrac{f(x)}{g(x)}$ in terms of partial fractions as,
$$\dfrac{f(x)}{g(x)}=\dfrac{A_{1}}{L_{1}}+\dfrac{A_{2}}{L_{2}(x)}+ \ldots + \dfrac{A_{m}}{L_{m}(x)}$$
where, $A_{i} \in \mathbb{R}$ constants.

\subsubsection{$g(x)$ is the product of non-repeating linear factors,\\ but a particular factor is repeated $k$ times}
Let $$g(x)=L_{1}^k(x) \cdot L_{2}(x) \cdot \ldots \cdot L_{\eta}(x)$$
\\Then, we can expand $\dfrac{f(x)}{g(x)}$ in terms of partial fractions as,

$$\dfrac{f(x)}{g(x)}=\dfrac{A_{1}}{L_{1}(x)}+\dfrac{A_{2}}{L_{1}^2(x)}+\dfrac{A_{3}}{L_{1}^3(x)}+\ldots + \dfrac{A_{k}}{L_{1}^k(x)}+\dfrac{B_{2}}{L_{2}(x)} \ldots +\dfrac{B_{\eta}}{L_{\eta}(x)}$$

\subsubsection{$g(x)$ contains some non-repeating linear as well as quadratic factors}
Let $$g(x)=\prod_{i} L_{i}(x) \cdot \prod_{j} Q_{j}(x)$$
where, $Q_{j}(x)$ are quadratic factors.

Then, we can expand $\dfrac{f(x)}{g(x)}$ in terms of partial fractions as,

$$\dfrac{f(x)}{g(x)}=\nsum[1.4]_{i} \dfrac{A_{i}}{L_{i}(x)}+ \nsum[1.4]_{j} \dfrac{xB_{j} +C_{j}}{Q_{j}(x)}$$

\subsubsection{$g(x)$ contains some non-repeating linear and repeating quadratic factors}
Let $$g(x)=\prod_{i} L_{i}(x) \prod_{j} Q_{j} (x) \prod_{\omega} Q_{\omega}^k (x)$$

Where, $Q_{\omega}^k(x)$ are repeating quadratic factors.\\
Then, we can expand $\dfrac{f(x)}{g(x)}$ in terms of partial fractions as,

$$\dfrac{f(x)}{g(x)}=\nsum[1.4]_{i} \dfrac{A_{i}}{L_{i}(x)}+ \nsum[1.4]_{j} \dfrac{x B_{j}+ C_{j}}{Q_{j}(x)}+\nsum[2]_{\omega} \nsum[1.4]_{r} \dfrac{xD_{r}+E_{r}}{Q_{\omega}^r (x)}$$

\subsection{Integrals of form}

\subsubsection{$$I=\displaystyle\int f\left(x+\dfrac{1}{x}\right)\left(1-\dfrac{1}{x^2}\right) \, dx$$}
Let, $u=x+\dfrac{1}{x} \implies du = \left(1-\dfrac{1}{x^2}\right) \, dx$
\\
Now, $I=\displaystyle\int f(u) \, du$

\subsubsection{Integrals of form $$\displaystyle\int \dfrac{x^2+1}{x^4+kx^2+1} \, dx$$}

Divide numerator and denominator by $x^2$
\\
Now, the transformed integral can be integrated using previous methods.

\subsection{Integration of Special Irrational Functions}
\subsubsection{Integrals of form }
$$\displaystyle\int \dfrac{1}{(ax+b)\sqrt{cx+d}} \, dx$$
Using $\textit{u-sub}$,
\\
Let $$u^2=cx+d$$
Now, the transformed integral can be integrated using previous methods.

\subsubsection{Integrals of form }
$$\displaystyle\int \dfrac{1}{(ax^2+bx+c) \sqrt{px+q}} \, dx$$
Using $\textit{u-sub}$,
\\
Let $$u^2=px+q$$
Now, the transformed integral can be integrated using previous methods.

\subsubsection{Integrals of form }
$$\displaystyle\int \dfrac{1}{(ax+b)(\sqrt{px^2+qx+r})} \, dx$$
Using $\textit{u-sub}$,
\\
Let $$\dfrac{1}{u}=ax+b$$
Now, the transformed integral can be integrated using previous methods.

\subsubsection{Integrals of form }
$$\displaystyle\int \dfrac{1}{(ax^2+b)\sqrt{cx^2+d}} \, dx$$
Using $\textit{u-sub}$,
\\
Let $$\dfrac{1}{\sqrt{u}}=x$$
Now, the transformed integral can be integrated using previous methods.

\subsubsection{Integrals of form }
$$\displaystyle\int \dfrac{1}{(x-k)^r\sqrt{ax^2+bx+c}} \, dx, r \ge 2$$
Using $\textit{u-sub}$,
\\
Let $$\dfrac{1}{u}=x-k$$
Now, the transformed integral can be integrated using previous methods.

\subsubsection{Integrals of form }
$$\displaystyle\int \dfrac{ax^2+bx+c}{(\alpha x + \beta)\sqrt{px^2+qx+r}} \, dx$$

Express, $$ax^2+bx+c=\lambda (\alpha x + \beta)\left[\dv{x}(px^2+qx+r)\right]+\mu (\alpha x + \beta)+\gamma$$
Now, the transformed integral can be integrated using previous methods.

\subsection{integrals of form }
$$\displaystyle\int \sin^m x \cdot \cos^n x \, dx$$
\subsubsection{If one of $m$ or $n$ is odd, $\hspace{2mm} m,n \in \mathbb{N}$ }
Then, we \textit{u-sub} the term with even power, i.e.
\\
If $m=2k+1, n=2p$ then, $u=\cos x$ \\
If $m=2p, n=2k+1$ then, $u=\sin x$
\subsubsection{If both $m$ and $n$ are odd, $\hspace{2mm} m,n \in \mathbb{N}$}
Then, \textit{u-sub} any of $\cos x$ or $\sin x$.
\subsubsection{If both $m$ and $n$ are even, $\hspace{2mm} m,n \in \mathbb{N}$}
Use Trig. indentities
\subsubsection{If $\dfrac{m+n-2}{2} \in \mathbb{Z^-}$, $\hspace{2mm} m,n \in \mathbb{Q}$}
Then, \textit{u-sub}, $u=\tan x$ or $u=\cot x$
\subsection{Integrals of form }
$$\displaystyle\int x^m (a+bx^n)^p \, dx$$
\subsubsection{If $P \in \mathbb{N}$}
Use binomial expansion and then integrate.
\subsubsection{If $P \in \mathbb{Z^-}$}
Use \textit{u-sub}, 
\\
Let $$u^k=x, \hspace{2mm} k = LCM(m,n)$$
\subsubsection{If $\dfrac{m+1}{n} \in \mathbb{Z}$ and $P \in \mathbb{Q}$}
Use \textit{u-sub}, 
\\
Let $$u^k=a+bx^n$$
where, if $P=\dfrac{a}{b}, HCF(a,b)=1$ then $k=b$
\subsubsection{$\dfrac{m+1}{n}+P \in \mathbb{Z}$ and $P \in \mathbb{Q}$}
Use \textit{u-sub}, 
\\
Let $$u^kx^n=a+bx^n$$
where, if $P=\dfrac{a}{b}, HCF(a,b)=1$ then $k=b$

\end{document}