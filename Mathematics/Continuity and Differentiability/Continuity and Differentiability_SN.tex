\documentclass{article}
\usepackage{graphicx}
\usepackage{amsmath}
\usepackage{amssymb}
\usepackage[italicdiff]{physics}
\usepackage{enumerate}
\usepackage{microtype}
\DisableLigatures{encoding= *, family=*}
\usepackage{titlesec}
\usepackage{xfrac}
\setcounter{secnumdepth}{4}

\titleformat{\paragraph}
{\normalfont\normalsize\bfseries}{\theparagraph}{1em}{}
\titlespacing*{\paragraph}
{0pt}{3.25ex plus 1ex minus .2ex}{1.5ex plus .2ex}

\title{Continuity and Differentiability}
\author{}
\date{}

\begin{document}
\maketitle
\part*{Continuity}

\section{Continuity of a Function at a Point}
A function $f(x)$ is said to be continuous at $x=a$; where $a \in \text{domain of $f(x)$}$ \\
Iff $$\lim\limits_{x \to a^-}{f(x)}=\lim\limits_{x \to a^+}{f(x)}=f(a)$$

\section{Continuity at End Points}
Let $f(x)$ be defined on $[a,b]$ then 
\\
$f(x)$ is continuous at $x=a$ 
\\
Iff $$\lim\limits_{x \to a^+}{f(x)}=f(a)$$
\\
$f(x)$ is continuous at $x=b$
\\
Iff $$\lim\limits_{x \to b^-}{f(x)}=f(b)$$

\section{Discontinuity of a Function}
A function $f$ is discontinuous if it has any kind of "Breaks".

\subsection{Removable Discontinuity}
$\lim\limits_{x \to a}{f(x)}$ exists but it is not equal to $f(a)$ or $f(a)$ is not defined.
\subsubsection{Missing Point Discontinuity}
$\lim\limits_{x \to a }{f(x)}$ exists but $f(a)$ is not defined.

\subsubsection{Isolated Point Discontinuity}
$\lim\limits_{x \to a}{f(x)}$ exists but it is not equal to $f(a)$

\subsection{Non-Removable Discontinuity}
$\lim\limits_{x \to a}{f(x)}$ does not exist.

\subsubsection{Finite Discontinuity (Jump Discontinuity)}
$\lim\limits_{x \to a^-}{f(x)}=L_{1}, \lim\limits_{x \to a^+}{f(x)}=L_{2},$ 
$L_{1} \not= L_{2}, L_{1}$ and $L_{2}$ are finite.
\paragraph{Jump of Discontinuity}
$|L_{1}-L_{2}|$ is the Jump of Discontinuity

\paragraph{Piecewise Continuous or Sectionally continuous Function}
A function having a finite number of jumps in a given interval.

\subsubsection{Infinite Discontinuity}
$\lim\limits_{x \to a^-}{f(x)}=L_{1}$ and $\lim\limits_{x \to a^+}{f(x)}=L_{2}$. Either $L_{1}$ or $L_{2}$ is $\pm \infty$
\newline \newline
For graph of $y=f(x)$, if at $x=a$ there is a vertical Asymptote, then there is a Infinite Discontinuity at $x=a$

\subsubsection{Oscillatory Discontinuity}
$\lim\limits_{x \to a}{f(x)}$ doesn't exist but oscillates between two finite quantites.

e.g. $f(x)=\sin \frac{1}{x}$

\section{Theorems on Continuity}
\subsection{Theorem 1}
Sum, difference and product of two continuous functions is always a continuous function. \newline 
However, quotient $h(x)=\frac{f(x)}{g(x)}$ is continuous at $x=a$ only if $g(a) \not= 0$.

\subsection{Theorem 2}
If $f(x)$ is continuous and $g(x)$ is discontinuous at $x=a$, then the product function $\phi (x)=f(x) \cdot g(x)$is not necessarily be discontinuous at $x=a$.

\subsection{Theorem 3}
If $f(x)$ and $g(x)$ both are discontinuous at $x=a$, then the product function $\phi (x)=f(x) \cdot g(x)$ is not necessarily be discontinuous at $x=a$.

\section{Continuity of Composite Functions}
Let $f(x)$ and $g(x)$ be two discontinuous functions and $g(x)$ is discontinuous $\forall \hspace{2mm} x \in [\alpha, \beta]$ 
$\left\{\right\}$
$$On$$ $$ Hold$$

\section{Intermediate Value Theorem}
If $f(x)$ is continuous $\hspace{1mm} \forall \hspace{2mm} x \in [\alpha,\beta]$ and $f(a) \not= f(b)$, then for any value\\ $L \in (f(a),f(b))$, there exists atleast one number $c \in (a,b)$ for which $f(c)=L$

\section{Continuity of Rational and \\Irrational Functions}
Consider the function,
$f:\mathbb{R} \rightarrow \mathbb{R}$
$$f(x)=\begin{cases}
    f_{1}(x) & x\in \mathbb{Q} \\
    f_{2}(x) & x\notin \mathbb{Q}
\end{cases}$$ \\
then $f$ is continous at $x \in A$ where $A=\left\{c:f_{1}(c)=f_{2}(c)\right\}$

\part*{Differentiability}
\section{Meaning of a Derivative}
\begin{itemize}
    \item The instantaneous rate of change of a function w.r.t. the independent variable is called the $Derivative$.
    \item Derivative also represents the slope of tangent line on a curve.
    \item Derivative of a function $f$ is generally denoted by $$f'(x), \dv{x}(f(x)), \dv{f(x)}{x}, \dv{x}f(x)$$
    \item Fundamental Definition of a Derivative $$f(x)=\lim\limits_{\Delta x \to 0}{\dfrac{f(x+\Delta x)-f(x)}{\Delta x}}$$
    \item Evalution of Derivative at a point (Say $a$) is denoted by $$f'(a), \dv{x}(f(x)) \biggr\rvert_{x=a}$$
\end{itemize}

\section{Existance of a Derivative}
The derivative of a function $f$ exists at $x=a$,\\ iff, $$\lim\limits_{h \to 0}{\dfrac{f(x+h)-f(x)}{h}}$$ exists and is finite
\\ Derivative doesn't exist at sharp points.

\subsection{Right Hand Derivative}
The Right Hand Derivative of $f(x)$ at $x=a$ is 
$$f'(a^+)=\lim\limits_{h \to 0^+}{\frac{f(a+h)-f(a)}{h}}$$

\subsection{Left Hand Derivative}
The Left Hand Derivative of $f(x)$ at $x=a$ is 
$$f'(a^-)=\lim\limits_{h \to 0^-}{\dfrac{f(a+h)-f(a)}{h}}$$

\section{Relation between Continuity and \\Differentiability}
If a function is differentiable at a point, it is necessarily continuous at that point. But the converse is not necessarily true.

\section{Differentiability in an Interval}
\begin{itemize}
    \item A function $f$ defined in an open interval $(a,b)$ is said to be differentiable in open interval $(a,b)$, if it is differentiable at each point in $(a,b)$.
    \item A function $f$ defined in a close interval $[a,b]$ is said to differentiable at end points $a$ and $b$, if RHD at $a$ and LHD at $b$ both exist and are finite. 
\end{itemize}

\section{Theorems on Differentiability}
\subsection{Theorem 1}
If $f(x)$ and $g(x)$ are both differentiable at $x=a$, $f(x) \pm g(x), f(x) \cdot g(x)$ will also be differentiable at $x=a$ but $\dfrac{f(x)}{g(x)}$ is differentiable only at $x=a$ if $g(a) \not= 0$.

\subsection{Theorem 2}
If $f(x)$ is differentiable at $x=a$ and $g(x)$ is not differentiable at $x=a$, then $f(x) \pm g(x)$ will not be differentiable at $x=a$, 

However nothing can be said about the product function $f(x) \cdot g(x)$.

\subsection{Theorem 3}
If both $f(x)$ and $g(x)$ are not differentiable at $x=a$, then nothing can be said about the sum, difference, product function.

\subsection{Theorem 4}
If $f(x)$ is differentiable at $x=a$ and $f(a)$ and $g(x)$ is continous at $x=a$

Then, the product function $f(x) \cdot g(x)$ will be differentiable at $x=a$

\subsection{Theorem 5}
Derivative of a continous function need not be a continous 
\end{document}