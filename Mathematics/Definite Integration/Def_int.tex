\documentclass{article}
\usepackage{graphicx, amsmath, amssymb, enumerate, microtype, titlesec, xfrac, xcolor, mathtools, multicol, bigints}
\usepackage[italicdiff]{physics}
\usepackage[bookmarks=false]{hyperref}
\hypersetup{
    colorlinks=true,
    linkcolor=[RGB]{59 108 209},
    urlcolor=[RGB]{59 108 209}
}
\urlstyle{same}
\DisableLigatures{encoding= *, family=*}
\setcounter{secnumdepth}{4}
\newcommand{\nsum}[1][1.4]{% only for \displaystyle
    \mathop{%
        \raisebox
            {-#1\depthofsumsign+1\depthofsumsign}
            {\scalebox
                {#1}
                {$\displaystyle\sum$}%
            }
    }
}
\newlength{\depthofsumsign}
\setlength{\depthofsumsign}{\depthof{$\sum$}}
\newlength{\totalheightofsumsign}
\newlength{\heightanddepthofargument}
\def\tmp#1 #2\relax{#1}
\setbox0=\hbox{$\xdef\intfont{%
    \expandafter\tmp\fontname\textfont3\expandafter\space\space\relax}$}
\font\tmp=\intfont\space at10pt\relax
\setbox0=\hbox{$\textfont3=\tmp \displaystyle \int$}
\dimen0=\ht0 \advance\dimen0 by\dp0 \divide\dimen0 by10
\xdef\intsize{\the\dimen0}
\def\dividedimen (#1/#2){\expandafter\ignorept\the
   \dimexpr\numexpr\number\dimexpr#1\relax
   *65536/\number\dimexpr#2\relax\relax sp\relax
}
{\lccode`\?=`\p \lccode`\!=`\t  \lowercase{\gdef\ignorept#1?!{#1}}}
\def\flexibleint{\def\fxintL{}\def\fxintU{}\futurelet\next\fxintA}
\def\fxintA{\ifx\next_\expandafter\fxintB\else\expandafter\fxintC\fi}
\def\fxintB_#1{\def\fxintL{#1}\fxintC}
\def\fxintC{\futurelet\next\fxintD}
\def\fxintD{\ifx\next^\expandafter\fxintE\else\expandafter\fxintF\fi}
\def\fxintE^#1{\def\fxintU{#1}\fxintF}
\def\fxintF#1{\begingroup
   \setbox0=\hbox{$\displaystyle{#1}$}%
   \dimen0=\ht0 \advance\dimen0 by\dp0
   \setbox1=\hbox{$\vcenter{\copy0}$}%
   \font\tmp=\intfont\space at\dividedimen(\dimen0/\intsize)pt
   \lower\dimexpr\dp0-\dp1\hbox{%
      $\textfont3=\tmp \displaystyle\int_{\fxintL}^{\fxintU}$}
   \box0
   \endgroup
}
\newcommand*\fullcirc[1][0.3ex]{\tikz\fill (0,0) circle (#1);}
\titleformat{\paragraph}
{\normalfont\normalsize\bfseries}{\theparagraph}{1em}{}
\titlespacing*{\paragraph}
{0pt}{3.25ex plus 1ex minus .2ex}{1.5ex plus .2ex}
\title{Definite Integration}
\author{}
\date{}

\begin{document}
\maketitle

\section{Definite Integral}
Let $f(x)$ be a function defined in the closed interval $[a,b]$ and $F(x)$ be its anti-derivative, then $$\displaystyle\int_{a}^{b} f(x) \, dx=F(b)-F(a) $$
is called the definite integral of the function $f(x)$ over the interval $[a,b]$, $a$ and $b$ are called limits of integration, lower and upper limit respectively.

\section{Geometrical Interpretation of Definite Integral}
If $f(x)>0 \hspace{2mm} \forall \hspace{2mm} x \in [a,b]$, then $\displaystyle\int_{a}^{b} f(x) \, dx $ is numerically equal to the area bounded by the curves $y=f(x), y=0, x=a, x=b$
\\
In general, $\displaystyle\int_{a}^{b} f(x) \, dx $ represents the net signed area (or algebraic sum of areas) i.e area below the axis of $x$ is counted as $-ve$ and that above is counted as $+ve$

\section{Definite Integration by \textit{u-sub}}
To evaluate definite integral of type, $$I=\displaystyle\int_{a}^{b} f(x)g'(x) \, dx $$
Let $$u=g(x) \implies du=g'(x) \, dx$$
Now, $I$ trasforms to, $$I=\displaystyle\int_{g(a)}^{g(b)} f(u) \, du $$
\subsection*{Important Note}
\begin{itemize}
    \item For the substitution to be valid, it must be continuous in the interval of integration, i.e. If $u=g(x)$, then $g(x)$ must be continuous in $[a,b]$.
\end{itemize}

\section{Properties of Definite Integration}
\subsection{Definite Integration is independent of the change of variable}
$$\displaystyle\int_{a}^{b} f(x) \, dx =\displaystyle\int_{a}^{b} f(u) \, du $$
\subsection{If limits of definite integral are flipped, then its value only differs in sign}
$$\displaystyle\int_{a}^{b} f(x) \, dx = - \displaystyle\int_{b}^{a} f(x) \, dx $$
\section{King's Rule}
$$\displaystyle\int_{a}^{b} f(x)  \, dx=\displaystyle\int_{a}^{b} f(a+b-x) \, dx  $$
\section{Integration of Piecewise Functions }
\subsection{}
$$\displaystyle\int_{a}^{b} f(x) \, dx=\displaystyle\int_{a}^{c} f(x) \, dx +\displaystyle\int_{c}^{b} f(x) \, dx  $$
$\hfill c \in \mathbb{R}$
\subsection{}
$$\displaystyle\int_{0}^{a} f(x) \, dx=\displaystyle\int_{0}^{\sfrac{a}{2} } f(x) \, dx +\displaystyle\int_{0}^{\sfrac{a}{2} } f(a-x) \, dx  $$
\section{Integration of Even, Odd Functions}
$$\displaystyle\int_{-a}^{a} f(x) \, dx =\begin{cases}
        0                                      & f(x)=-f(-x) \text{ Odd symmetric about }
        x= 0                                                                                  \\
        2 \displaystyle\int_{0}^{a} f(x) \, dx & f(x)=f(-x) \text{ Even symmetric about } x=0\end{cases}$$
\section{Integration in case of Even, Odd Symmetries}
$$\displaystyle\int_{a}^{b} f(x) \, dx = \begin{cases}

        0                                                      & f(a+x)=-f(b-x) \hspace{23mm}\text{ Odd symmetric} \\[2mm] & \text{or } f\left(\dfrac{a+b}{2}-x \right)=-f\left(\dfrac{a+b}{2}+x \right) { \text{about} } \hspace{1mm}x=\dfrac{a+b}{2}\\[3mm]
        2 \displaystyle\int_{\sfrac{(a+b)}{2}}^{b} f(x)  \, dx & f(a+x)=f(b-x) \hspace{23mm}\text{ Even symmetric} \\[2mm] & \text{or } f\left(\dfrac{a+b}{2}-x \right)=f\left(\dfrac{a+b}{2}+x \right) { \text{about} } \hspace{1mm}x=\dfrac{a+b}{2}\end{cases}$$

\section{Change of Limits property}
$$\displaystyle\int_{a}^{b} f(x) \, dx =\left(b-a\right)\displaystyle\int_{0}^{1} f[(b-a)x+a] \, dx $$
Now, we can add any two (or more) integrals as they all will have same limits $0$ and $1$.
\section{Even and Odd Integral Functions}
\subsection{Even Integral Functions}
If $O(x)$ is Odd Symmetric about $x=0$ \\[2mm]
then, $f(x)=\displaystyle\int_{a}^{x} O(x) \, dx $ is Even Symmetric about $x=0$
\subsection{Odd Integral Functions}
If $E(x)$ is Even Symmetric about $x=0$ and $\displaystyle\int_{0}^{a} E(x) \, dx =0$\\[2mm]
then, $f(x)=\displaystyle\int_{a}^{x} E(x) \, dx $ is Odd symmetric about $x=0$
\section{Integration of Periodic Functions}
Let $f(x)$ be a function with fundamental period $T$
\subsection{Area under a periodic curve repeats with \\ same periodicity}
$$\displaystyle\int_{0}^{nT} f(x) \, dx =n\displaystyle\int_{0}^{T} f(x) \, dx $$
\subsection{Area under a periodic curve over an interval with length same as that of the period is same as the area enclosed in first period}
$$\displaystyle\int_{a}^{a+T} f(x) \, dx=\displaystyle\int_{0}^{T}f(x)  \, dx  $$
Also, \begin{enumerate}[i.]
    \item $$\displaystyle\int_{a}^{a+nT} f(x) \, dx=\displaystyle\int_{0}^{nT} f(x) \, dx =n\displaystyle\int_{0}^{T} f(x) \, dx $$
    \item $$\displaystyle\int_{a+nT}^{b+nT} f(x) \, dx= \displaystyle\int_{a}^{b} f(x) \, dx $$
    \item $$\displaystyle\int_{a}^{b+nT} f(x) \, dx =\displaystyle\int_{a}^{b} f(x) \, dx + n \displaystyle\int_{0}^{T} f(x) \, dx $$
    \item $$\displaystyle\int_{n_{1}T}^{n_{2}T} f(x) \, dx =(n_{2}-n_{1})\displaystyle\int_{0}^{T} f(x)  \, dx $$
    \item $$\displaystyle\int_{a+n_{1}T}^{b+n_{2}T} f(x)  \, dx =\displaystyle\int_{a}^{b} f(x) \, dx +(n_{2}-n_{1})\displaystyle\int_{0}^{T} f(x) \, dx $$
\end{enumerate}
\section{Newton-Leibnitz's Rule for Differentiation \\under Integral sign}
\subsection{General Case}
If $\Phi (x) $ and $\Psi (x)$ are defined on $[a,b]$and are diffenetiable on $(a,b)$ and $f(x,t)$ is continuous, then
$$\dv{x}\left[\displaystyle\int_{\Phi (x)}^{\Psi (x)} f(x,t) \, dt \right]=\displaystyle\bigintsss_{\Phi (x)}^{\Psi (x)} \pdv{x}f(x,t) \, dt \hspace{1.5mm}+\Psi'(x)f(x,\Psi(x))+\Phi'(x)f(x,\Phi(x)) $$

\subsection*{Corollary 1}
$$\dv{x}\left[\displaystyle\int_{\Phi (x)}^{\Psi (x)} f(t) \, dt \right]=\Psi'(x)f(\Psi(x))+\Phi'(x)f(\Phi(x)) $$
\subsection*{Corollary 2}
$$\dv{x}\left[\displaystyle\int_{a}^{b} f(x,t) \, dt \right]=\displaystyle\bigintssss_{a}^{b} \pdv{x}f(x,t) \, dt $$
\section{Fundamental Definition of \\Definite Integration, \\ \textit{Integration as limit of a sum}}
\subsection{Definite Integral as limit of Riemann Sum}

$$\displaystyle\int_{a}^{b} f(x) \, dx = \lim_{n \to \infty}{\nsum[1.4]_{i=1}^{n} {f(x_{i}^*) \Delta x_{i}}}$$
\begin{itemize}
    \item The interval $[a,b]$ is divided into $n$ partitions.
    \item $a=x_{0}<x_{1}<x_{2}<x_{3}< \ldots < x_{n-1}<x_{n}=b$
    \item $x_{i}^*$ is a point lying in the interval $(x_{i-1},x_{i}) \hfill i \in \mathbb{Z^+}$
    \item $f(x^{*}_{i})$ gives height of formed rectangle, $\Delta x_{i} = x_{i}-x_{i-1}$ is its width.
\end{itemize}
\subsubsection{Definite Integral as limit of Upper Riemann Sum}
$$\displaystyle\int_{a}^{b} f(x) \, dx = \lim_{n \to \infty}{\nsum[1.4]_{i=1}^{n} {f(x_{i}^*) \Delta x}}$$
$f(x^*_{i})=max \left\{f(x_{i-1}),f(x_{i})\right\}$
\subsubsection{Definite Integral as limit of Lower Riemann Sum}
$$\displaystyle\int_{a}^{b} f(x) \, dx = \lim_{n \to \infty}{\nsum[1.4]_{i=1}^{n} {f(x_{i}^*) \Delta x}}$$
$f(x^*_{i})=min \left\{f(x_{i-1}),f(x_{i})\right\}$
\subsubsection{Definite Integral as limit of Left Endpoint Riemann Sum}
$$\displaystyle\int_{a}^{b} f(x) \, dx = \lim_{n \to \infty}{\nsum[1.4]_{i=1}^{n} {f(x_{i}^*) \Delta x}}$$
\begin{enumerate}[i.]
    \item $x^*_{i}=x_{i-1}=\dfrac{(b-a)}{n}(i-1) $
    \item $x_{0}=x_{1}=x_{2}= \ldots = x_{n-1}=x_{n}$
    \item $\Delta x= \dfrac{b-a}{n} $

\end{enumerate}
\subsubsection{Definite Integral as limit of Midpoint Riemann Sum}
$$\displaystyle\int_{a}^{b} f(x) \, dx = \lim_{n \to \infty}{\nsum[1.4]_{i=1}^{n} {f(x_{i}^*) \Delta x}}$$
\begin{enumerate}[i.]
    \item $x^*_{i}=\dfrac{x_{i-1}+x_{i}}{2}= \dfrac{\left(b-a\right)}{n}(2i-1)$
    \item $x_{0}=x_{1}=x_{2}= \ldots = x_{n-1}=x_{n}$
    \item $\Delta x= \dfrac{b-a}{n} $
\end{enumerate}
\subsubsection{Definite Integral as limit of Right Endpoint Riemann Sum}
$$\displaystyle\int_{a}^{b} f(x) \, dx = \lim_{n \to \infty}{\nsum[1.4]_{i=1}^{n} {f(x_{i}^*) \Delta x}}$$
\begin{enumerate}[i.]
    \item $x^*_{i}=x_{i}=\dfrac{(b-a)}{n}i$
    \item $x_{0}=x_{1}=x_{2}= \ldots = x_{n-1}=x_{n}$
    \item $\Delta x= \dfrac{b-a}{n} $
\end{enumerate}
\subsection{$\dfrac{r}{n} $ form}
$$\lim\limits_{n \to \infty}{\dfrac{1}{n} \nsum[1.4]_{r=\Phi (n)}^{\Psi (n)} f\left(\dfrac{r}{n}\right) }=\displaystyle\int_{a}^{b} f(x) \, dx $$
$a= \lim\limits_{n \to \infty}{\dfrac{\Phi (n)}{n} }, b = \lim\limits_{n \to \infty}{ \dfrac{\Psi (n)}{n} }$

\section{Inequalities involving Definite Integrals}
\subsection{}
$$\abs{\displaystyle\int_{a}^{b} f(x) \, dx }\le \displaystyle\int_{a}^{b} \abs{f(x)} \, dx $$
Equality holds when $f(x)$ is of same sign $\forall \hspace{1mm} x \in [a,b]$
\subsection{}
$$\abs{\displaystyle\int_{a}^{b} f(x)g(x) \, dx }\le \sqrt{\left(\displaystyle\int_{a}^{b} f^2(x) \, dx \right)\left(\displaystyle\int_{a}^{b} g^2(x) \, dx \right)}$$
\subsection{}
If $f(x)$ is continuous on $[a,b]$ and $f_{l}(x)$ and $f_{h}(x)$ are also continuous on $[a,b]$ such that,
$$f_{l}(x) \le f(x) \le f_{h}(x)$$
$\hspace{11cm}\forall \hspace{1mm} x \in [a,b]$
then, $$\displaystyle\int_{a}^{b} f_{l}(x) \, dx \le \displaystyle\int_{a}^{b} f(x) \, dx \le \displaystyle\int_{a}^{b} f_{h}(x) \, dx $$
\subsection{}
If $m$ and $M$ be global minimum and global maximum of $f(x)$ respectively in $[a,b]$, then
$$m(b-a) \le \displaystyle\int_{a}^{b} f(x) \, dx \le M(b-a)$$
\section{Gamma Function}
Let $s >0 $, then the improper integral \begin{align*}
    \Gamma (s) &=\displaystyle\int_{0}^{\infty} e^{-t} \, t^{s-1} \, dt \\[2mm]
    &= \displaystyle\int_{0}^{1} \left(\ln \left(\dfrac{1}{t} \right)\right)^{s-1} \, dt 
\end{align*}
is defined to be the Gamma Function.
\subsection{Properties of Gamma Function}
\begin{enumerate}[i.]
    \item $\Gamma (s+1)=s \cdot \Gamma (s)$
    \item $\Gamma (n+1)= n! \hfill n \in \mathbb{N}_{0}$
    \item $\Gamma \left(\dfrac{1}{2}\right)= \sqrt{\pi}$
    \item $\displaystyle\int_{0}^{\sfrac{\pi}{2}}\sin^m x \cdot \cos^n x  \, dx = \dfrac{\Gamma \left(\dfrac{m+1}{2} \right) \Gamma \left(\dfrac{n+1}{2} \right)}{2 \Gamma \left(\dfrac{m+n+2}{2} \right)} $
    \item $\Gamma \left(s\right) \cdot \Gamma \left(1-s\right)= \dfrac{\pi}{\sin s \pi} \hfill s \in (0,1)$
    \item $\Gamma \left(s\right) \cdot \Gamma \left(s + \dfrac{1}{2} \right)= \dfrac{\sqrt{\pi}}{2^{2s-1}} \cdot \Gamma \left(2s\right)$
    \item $\displaystyle\prod_{r=1}^{n-1} \Gamma \left(\dfrac{r}{n} \right)=\dfrac{\left(2 \pi\right)^{\dfrac{n-1}{2} }}{\sqrt{n}} $
\end{enumerate}
\section{Beta Function}
Let $m,n >0$, then the integral \begin{equation*}
    \begin{split}
        B(m,n)&=\displaystyle\int_{0}^{1} t^{m-1}\left(1-t\right)^{n-1} \, dt\\[2mm]
        &=\displaystyle\int_{0}^{\infty} \dfrac{t^{m-1}}{\left(1+t\right)^{m+n}}  \, dt
    \end{split}
\end{equation*}
is defined to be the Beta Function.
\subsection{Properties of Beta Function}
\begin{enumerate}[i.]
    \item $B(m,n)=B(n,m)$
    \item $B(m,n)=\dfrac{\Gamma \left(m\right) \Gamma \left(n\right)}{\Gamma \left(m+n\right)} $
\end{enumerate}
\section{Walli's Formula}
\subsection{}
To evaluate integrals of form, $$\displaystyle\int_{0}^{\sfrac{\pi}{2} } \sin^m x \cdot \cos^n x \, dx $$
where, $m,n \in \mathbb{Z^+}$

    \begin{align*}
        \displaystyle\int_{0}^{\sfrac{\pi}{2}} \sin^m x \cdot \cos^n x  \, dx&=\displaystyle\int_{0}^{\sfrac{\pi}{2}} \sin^n x \cdot \cos^m x \, dx  \\[2mm]
        &=\dfrac{\displaystyle\prod_{r\ge 0} \left(m-2r-1\right) \cdot \displaystyle\prod_{r\ge 0} \left(n-2r-1\right)}{\displaystyle\prod_{r\ge 0}  \left(m+n-2r\right)} \cdot \dfrac{\pi}{2} \tag{$m,n \in {\left\{2k:k \in \mathbb{N}\right\}}$ }\\
        &= \dfrac{\displaystyle\prod_{r\ge 0}  \left(m-2r-1\right) \cdot \displaystyle\prod_{r\ge 0} \left(n-2r-1\right)}{\displaystyle\prod_{r\ge 0}  \left(m+n-2r\right)} \tag{Any one of $m,n$ is odd}
    \end{align*}
\subsection{}
\begin{align*}
    \displaystyle\int_{0}^{\sfrac{\pi}{2}} \sin^n x \, dx&=\displaystyle\int_{0}^{\sfrac{\pi}{2}}\cos^n x \, dx  \\[2mm]
    &= \dfrac{\pi}{2} \cdot  \displaystyle\prod_{r\ge 0} \dfrac{n-2r-1}{n-2r}  \tag*{$n \in \left\{2k:k\in \mathbb{N}\right\}$}\\
    &= \displaystyle\prod_{r\ge 0} \dfrac{n-2r-1}{n-2r} \tag*{$n \in \left\{2k-1:k\in \mathbb{N}\right\}$}
\end{align*}
\end{document}