\documentclass{article}
\usepackage{graphicx, amsmath, amssymb, enumerate, microtype, titlesec, xfrac, xcolor, mathtools, multicol}
\usepackage[italicdiff]{physics}
\usepackage[bookmarks=false]{hyperref}
\hypersetup{
    colorlinks=true,
    linkcolor=[RGB]{59 108 209},
    urlcolor=[RGB]{59 108 209}
}
\urlstyle{same}
\DisableLigatures{encoding= *, family=*}
\setcounter{secnumdepth}{4}
\newcommand{\nsum}[1][1.4]{% only for \displaystyle
    \mathop{%
        \raisebox
            {-#1\depthofsumsign+1\depthofsumsign}
            {\scalebox
                {#1}
                {$\displaystyle\sum$}%
            }
    }
}
\newlength{\depthofsumsign}
\setlength{\depthofsumsign}{\depthof{$\sum$}}
\newlength{\totalheightofsumsign}
\newlength{\heightanddepthofargument}
\def\tmp#1 #2\relax{#1}
\setbox0=\hbox{$\xdef\intfont{%
    \expandafter\tmp\fontname\textfont3\expandafter\space\space\relax}$}
\font\tmp=\intfont\space at10pt\relax
\setbox0=\hbox{$\textfont3=\tmp \displaystyle \int$}
\dimen0=\ht0 \advance\dimen0 by\dp0 \divide\dimen0 by10 
\xdef\intsize{\the\dimen0}
\def\dividedimen (#1/#2){\expandafter\ignorept\the
   \dimexpr\numexpr\number\dimexpr#1\relax
   *65536/\number\dimexpr#2\relax\relax sp\relax
}
{\lccode`\?=`\p \lccode`\!=`\t  \lowercase{\gdef\ignorept#1?!{#1}}}
\def\flexibleint{\def\fxintL{}\def\fxintU{}\futurelet\next\fxintA}
\def\fxintA{\ifx\next_\expandafter\fxintB\else\expandafter\fxintC\fi}
\def\fxintB_#1{\def\fxintL{#1}\fxintC}
\def\fxintC{\futurelet\next\fxintD}
\def\fxintD{\ifx\next^\expandafter\fxintE\else\expandafter\fxintF\fi}
\def\fxintE^#1{\def\fxintU{#1}\fxintF}
\def\fxintF#1{\begingroup
   \setbox0=\hbox{$\displaystyle{#1}$}%
   \dimen0=\ht0 \advance\dimen0 by\dp0
   \setbox1=\hbox{$\vcenter{\copy0}$}%
   \font\tmp=\intfont\space at\dividedimen(\dimen0/\intsize)pt
   \lower\dimexpr\dp0-\dp1\hbox{%
      $\textfont3=\tmp \displaystyle\int_{\fxintL}^{\fxintU}$}
   \box0
   \endgroup
}
\newcommand*\fullcirc[1][0.3ex]{\tikz\fill (0,0) circle (#1);} 
\titleformat{\paragraph}
{\normalfont\normalsize\bfseries}{\theparagraph}{1em}{}
\titlespacing*{\paragraph}
{0pt}{3.25ex plus 1ex minus .2ex}{1.5ex plus .2ex}
\title{Definite Integration}
\author{}
\date{}

\begin{document}
\maketitle

\section{Definite Integral}
Let $f(x)$ be a function defined in the closed interval $[a,b]$ and $F(x)$ be its anti-derivative, then $$\displaystyle\int_{a}^{b} f(x) \, dx=F(b)-F(a) $$
is called the definite integral of the function $f(x)$ over the interval $[a,b]$, $a$ and $b$ are called limits of integration, lower and upper limit respectively.

\section{Geometrical Interpretation of Definite Integral}
If $f(x)>0 \hspace{2mm} \forall \hspace{2mm} x \in [a,b]$, then $\displaystyle\int_{a}^{b} f(x) \, dx $ is numerically equal to the area bounded by the curves $y=f(x), y=0, x=a, x=b$
\\
In general, $\displaystyle\int_{a}^{b} f(x) \, dx $ represents the net signed area (or algebraic sum of areas) i.e area below the axis of $x$ is counted as $-ve$ and that above is counted as $+ve$

\section{Definite Integration by \textit{u-sub}}
To evaluate definite integral of type, $$I=\displaystyle\int_{a}^{b} f(x)g'(x) \, dx $$
Let $$u=g(x) \implies du=g'(x) \, dx$$
Now, $I$ trasforms to, $$I=\displaystyle\int_{g(a)}^{g(b)} f(u) \, du $$
\subsection*{Important Note}
\begin{itemize}
    \item For the substitution to be valid, it must be continuous in the interval of integration, i.e. If $u=g(x)$, then $g(x)$ must be continuous in $[a,b]$.
\end{itemize}

\section{Properties of Definite Integration}
\subsection{Definite Integration is independent of the change of variable}
$$\displaystyle\int_{a}^{b} f(x) \, dx =\displaystyle\int_{a}^{b} f(u) \, du $$
\subsection{If limits of definite integral are flipped, then its value only differs in sign}
$$\displaystyle\int_{a}^{b} f(x) \, dx = - \displaystyle\int_{b}^{a} f(x) \, dx $$
\section{King's Rule}
$$\displaystyle\int_{a}^{b} f(x)  \, dx=\displaystyle\int_{a}^{b} f(a+b-x) \, dx  $$
\section{Integration of Piecewise Functions }
\subsection{}
$$\displaystyle\int_{a}^{b} f(x) \, dx=\displaystyle\int_{a}^{c} f(x) \, dx +\displaystyle\int_{c}^{b} f(x) \, dx  $$
$\hfill c \in \mathbb{R}$
\subsection{}
$$\displaystyle\int_{0}^{a} f(x) \, dx=\displaystyle\int_{0}^{\sfrac{a}{2} } f(x) \, dx +\displaystyle\int_{0}^{\sfrac{a}{2} } f(a-x) \, dx  $$
\section{Integration of Even, Odd Functions}
$$\displaystyle\int_{-a}^{a} f(x) \, dx =\begin{cases}
        0                                      & f(x)=-f(-x) \text{ Odd symmetric about }
        x= 0                                                                                  \\
        2 \displaystyle\int_{0}^{a} f(x) \, dx & f(x)=f(-x) \text{ Even symmetric about } x=0\end{cases}$$
\section{Integration in case of Even, Odd Symmetries}
$$\displaystyle\int_{a}^{b} f(x) \, dx = \begin{cases}

        0                                                      & f(a+x)=-f(b-x) \hspace{23mm}\text{ Odd symmetric}  \\[2mm] & \text{or } f\left(\dfrac{a+b}{2}-x \right)=-f\left(\dfrac{a+b}{2}+x \right) { about } \hspace{1mm}x=\dfrac{a+b}{2}\\[3mm]
        2 \displaystyle\int_{a}^{\sfrac{(a+b)}{2}} f(x)  \, dx & f(a+x)=f(b-x) \hspace{23mm}\text{ Even symmetric}  \\[2mm] & \text{or } f\left(\dfrac{a+b}{2}-x \right)=f\left(\dfrac{a+b}{2}+x \right) { about } \hspace{1mm}x=\dfrac{a+b}{2}\end{cases}$$

\end{document}