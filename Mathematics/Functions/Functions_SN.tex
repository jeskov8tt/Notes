\documentclass{article}
\usepackage{graphicx}
\usepackage{amsmath}
\usepackage{amssymb}
\usepackage[italicdiff]{physics}
\usepackage{enumerate}
\usepackage{microtype}
\DisableLigatures{encoding= *, family=*}
\usepackage{titlesec}
\usepackage{xfrac}
\setcounter{secnumdepth}{4}

\titleformat{\paragraph}
{\normalfont\normalsize\bfseries}{\theparagraph}{1em}{}
\titlespacing*{\paragraph}
{0pt}{3.25ex plus 1ex minus .2ex}{1.5ex plus .2ex}

\title{Functions}
\author{}
\date{}

\begin{document}
\maketitle

\section{Function}
\begin{itemize}
    \item Let $A$ and $B$ be two non-empty sets and $f$ be a relation which associates each element of set $A$ with unique element of set $B$, $f$ is called a function from $A$ to $B$.
    \item If any line $\parallel$ to axis of $y$ intersects graph of $y=f(x)$ at more than once then, $f$ is not a function.
\end{itemize}

\subsection{Domain}
Let $f:A \rightarrow B$ be a function then set $A$ which consits all those elements for which image under $f$ is well-defined, is called $Domain$ of $f$.

\subsubsection{Suggestions on finding domain of $f$}
\begin{enumerate}[i.]
    \item If $f(x)=\sqrt[2n]{g(x)}$ then Dom $f$ $\in g(x) \geq 0$
    \item If $f(x)=\dfrac{1}{g(x)}$ then Dom $f$ $\in \mathbb{R}-\left\{x:x=a, g(a)=0\right\}$
    \item If $f(x)=\log_{h(x)}{g(x)}$ then Dom $f$ $\in \left\{x:x=a,g(a)> 0\right\} \bigcap \left\{x:x=a, h(a)>0 \lor h(a) \not= 1\right\}$
    \item If $f(x)=g(x) + h(x)$ then Dom $f$ $\in \text{Dom} \hspace{1mm} g \bigcap \text{Dom} \hspace{1mm} h$
\end{enumerate}
\subsection{Co-domain}
Set $B$ is Co-domain of $f$.
\subsection{Range}
Range of $f$ is set of images of elements in domain $A$ under $f$.

\section{Periodic Functions}
A function $f(x)$ is said to be periodic if there exist a +ve real number $\lambda$ such that $$f(x+\lambda)=f(x)$$ The smallest of all such $\lambda$ is called the fundamental periodic or "periodic" of $f$.

\subsection{Period of some Standard functions}
\begin{tabular}{c c}
    Function                                     & Periodic                                      \\
    $\sin ^n x, \cos ^n x, \sec ^n x, \csc ^n x$ & $\begin{cases}
                                                            \pi  & n \in even                            \\
                                                            2\pi & n \in odd \hspace{1mm} \lor  fraction
                                                        \end{cases}$ \\
    $\tan ^n x, \cot ^n x$                       & $\pi n \in $ even $\lor$ odd                  \\
    $\abs{\sin x},\abs{\cos x}, \abs{\tan x}, \abs{\cot x}, \abs{\sec x}, \abs{\csc x}$ & $\pi$
\end{tabular}
\section{Odd and Even Functions}
\subsection*{Odd Functions}
A function $f$ is odd if, $$f(-x)=-f(x)$$ i.e. symmetric about origin.

\subsection*{Even Functions}
A function $f$ is even if, $$f(x)=f(-x)$$ i.e. symmetric about axis of $y$.

\subsection*{Properties}
\begin{enumerate}[i.]
    \item Product of two even or two odd is even.
    \item Product of odd and even is odd.
    \item Every function can be expressed as sum of a odd and even function.
    \item Derivative of odd is even and Derivative of even is odd.
    \item Even function or odd function when squared becomes even.
    \item Only function which is both even and odd is $f(x)=0$
\end{enumerate}
\end{document}