\documentclass{article}
\usepackage{graphicx, amsmath, amssymb, enumerate, microtype, titlesec, xfrac, xcolor, mathtools, multicol, mathrsfs}
\usepackage[italicdiff]{physics}
\usepackage[bookmarks=false]{hyperref}
\hypersetup{
    colorlinks=true,
    linkcolor=[RGB]{59 108 209},
    urlcolor=[RGB]{59 108 209}
}
\urlstyle{same}
\DisableLigatures{encoding= *, family=*}
\setcounter{secnumdepth}{4}
\newcommand{\nsum}[1][1.4]{% only for \displaystyle
    \mathop{%
        \raisebox
            {-#1\depthofsumsign+1\depthofsumsign}
            {\scalebox
                {#1}
                {$\displaystyle\sum$}%
            }
    }
}
\newlength{\depthofsumsign}
\setlength{\depthofsumsign}{\depthof{$\sum$}}
\newlength{\totalheightofsumsign}
\newlength{\heightanddepthofargument}
\def\tmp#1 #2\relax{#1}
\setbox0=\hbox{$\xdef\intfont{%
    \expandafter\tmp\fontname\textfont3\expandafter\space\space\relax}$}
\font\tmp=\intfont\space at10pt\relax
\setbox0=\hbox{$\textfont3=\tmp \displaystyle \int$}
\dimen0=\ht0 \advance\dimen0 by\dp0 \divide\dimen0 by10 
\xdef\intsize{\the\dimen0}
\def\dividedimen (#1/#2){\expandafter\ignorept\the
   \dimexpr\numexpr\number\dimexpr#1\relax
   *65536/\number\dimexpr#2\relax\relax sp\relax
}
{\lccode`\?=`\p \lccode`\!=`\t  \lowercase{\gdef\ignorept#1?!{#1}}}
\def\flexibleint{\def\fxintL{}\def\fxintU{}\futurelet\next\fxintA}
\def\fxintA{\ifx\next_\expandafter\fxintB\else\expandafter\fxintC\fi}
\def\fxintB_#1{\def\fxintL{#1}\fxintC}
\def\fxintC{\futurelet\next\fxintD}
\def\fxintD{\ifx\next^\expandafter\fxintE\else\expandafter\fxintF\fi}
\def\fxintE^#1{\def\fxintU{#1}\fxintF}
\def\fxintF#1{\begingroup
   \setbox0=\hbox{$\displaystyle{#1}$}%
   \dimen0=\ht0 \advance\dimen0 by\dp0
   \setbox1=\hbox{$\vcenter{\copy0}$}%
   \font\tmp=\intfont\space at\dividedimen(\dimen0/\intsize)pt
   \lower\dimexpr\dp0-\dp1\hbox{%
      $\textfont3=\tmp \displaystyle\int_{\fxintL}^{\fxintU}$}
   \box0
   \endgroup
}
\newcommand*\fullcirc[1][0.3ex]{\tikz\fill (0,0) circle (#1);} 
\titleformat{\paragraph}
{\normalfont\normalsize\bfseries}{\theparagraph}{1em}{}
\titlespacing*{\paragraph}
{0pt}{3.25ex plus 1ex minus .2ex}{1.5ex plus .2ex}
\title{Extra Topics}
\author{}
\date{}

\begin{document}
\maketitle

\section{Modulo Operator (Arithmetic Remainder)}
If $x \in \mathbb{R}^+$ and $n \in \mathbb{N}$, we can uniquely write $x=mn+r$, where $m \in \mathbb{W}$ and $r \in [0,n)$.
\newline
We define $$x \bmod n=r$$
\newline
e.g. $10.5 \bmod 3=1.5$
\section{Every Function can be expressed as sum of two Even and Odd Symmetric Functions about $x=a$}
Let $f(x)$ be any general function.\\
Let $E(x)$ be a function Even Symmetric about $x=a$ and\\ $O(x)$ be a function Odd Symmetric about $x=a$\\
$\therefore$
\begin{equation*}
    \begin{split}
    E(a+x)=E(a-x)\\
    O(a+x)=-O(a-x)
    \end{split}
\end{equation*}
such that, $$f(x)=E(x)+O(x)$$
Hence, 
\begin{equation*}
    \begin{split}
        E(x)=\dfrac{f(x)+f(2a-x)}{2}\\
        O(x)=\dfrac{f(x)-f(2a-x)}{2}  
    \end{split}
\end{equation*}

$$f(x)=\underbrace{\dfrac{f(x)+f(2a-x)}{2}}_{\text{Even Symmetric Part}}+\underbrace{\dfrac{f(x)-f(2a-x)}{2} }_{\text{Odd Symmetric Part}}$$
\end{document}