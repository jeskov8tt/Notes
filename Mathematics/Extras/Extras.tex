\documentclass{article}
\usepackage{graphicx}
\usepackage{amsmath}
\usepackage{amssymb}
\usepackage[italicdiff]{physics}
\usepackage{enumerate}
\usepackage{microtype}
\DisableLigatures{encoding= *, family=*}
\usepackage{titlesec}
\usepackage{xfrac}
\setcounter{secnumdepth}{4}

\titleformat{\paragraph}
{\normalfont\normalsize\bfseries}{\theparagraph}{1em}{}
\titlespacing*{\paragraph}
{0pt}{3.25ex plus 1ex minus .2ex}{1.5ex plus .2ex}

\title{}
\author{}
\date{}

\begin{document}
\maketitle

\section{Modulo Operator (Arithmetic Remainder)}
If $x \in \mathbb{R}^+$ and $n \in \mathbb{N}$, we can uniquely write $x=mn+r$, where $m \in \mathbb{W}$ and $r \in [0,n)$.
\newline
We define $$x \bmod n=r$$
\newline
e.g. $10.5 \bmod 3=1.5$
\end{document}