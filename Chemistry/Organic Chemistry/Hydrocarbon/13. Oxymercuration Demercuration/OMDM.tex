\documentclass{article}
\usepackage{graphicx}
\usepackage{amsmath}
\usepackage{amssymb}
\usepackage[italicdiff]{physics}
\usepackage{enumerate}
\usepackage{microtype}
\DisableLigatures{encoding= *, family=*}
\usepackage{titlesec}
\usepackage{xfrac}
\setcounter{secnumdepth}{4}
\usepackage{xcolor}
\usepackage[bookmarks=false]{hyperref}
\usepackage{mathtools}
\usepackage{tikz} 
\newcommand*\fullcirc[1][0.3ex]{\tikz\fill (0,0) circle (#1);} 
\usepackage{bigints}
\hypersetup{
    colorlinks=true,
    linkcolor=[RGB]{59 108 209},
    urlcolor=[RGB]{59 108 209}
}
\urlstyle{same}

\titleformat{\paragraph}
{\normalfont\normalsize\bfseries}{\theparagraph}{1em}{}
\titlespacing*{\paragraph}
{0pt}{3.25ex plus 1ex minus .2ex}{1.5ex plus .2ex}

\title{Oxymercuration and Demercuration}
\author{}
\date{}

\begin{document}
\maketitle

\section{Reaction and Mechanism}
\begin{center}
    \includegraphics[scale=0.22]{OMDM_1722369977375.JPEG}
\end{center}
\section{Reaction Observations}
    \begin{enumerate}[i.]
        \item Markonikov addition.
        \item No rearrangment.
        \item Both Syn and anti addition.
        \item OM is anti additon, DM involves $C^{\fullcirc}$, hence both syn and anti addition.
        \item Metal $Hg^{\fullcirc}$ is obtained.
    \end{enumerate}
\end{document}