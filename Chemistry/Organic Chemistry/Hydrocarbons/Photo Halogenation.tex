\documentclass{article}
\usepackage{graphicx}
\usepackage{amsmath}
\usepackage{amssymb}
\usepackage[italicdiff]{physics}
\usepackage{enumerate}
\usepackage{microtype}
\DisableLigatures{encoding= *, family=*}
\usepackage{titlesec}
\usepackage{xfrac}
\setcounter{secnumdepth}{4}
\usepackage{xcolor}
\usepackage[bookmarks=false]{hyperref}
\usepackage{mathtools}
\usepackage{bigints}
\usepackage{chemfig}
\hypersetup{
    colorlinks=true,
    linkcolor=[RGB]{59 108 209},
    urlcolor=[RGB]{59 108 209}
}
\urlstyle{same}

\titleformat{\paragraph}
{\normalfont\normalsize\bfseries}{\theparagraph}{1em}{}
\titlespacing*{\paragraph}
{0pt}{3.25ex plus 1ex minus .2ex}{1.5ex plus .2ex}

\title{}
\author{}
\date{}

\begin{document}
\maketitle

\section{Fluorination}
$$CH_{4} \hspace{5mm} + \hspace{5mm} F_{2} \hspace{3mm} \xrightarrow{\hspace{2mm} h \nu \hspace{2mm}} \hspace{3mm} C_{black} \hspace{5mm} + \hspace{5mm} 4 HF$$

$F$ is highly reactive, hence the reaction is exothermic.

\section{Iodination}
$$CH_{4}\hspace{5mm}+\hspace{5mm}I_{2}\hspace{3mm}\xrightleftharpoons[]{\hspace{2mm} h \nu \hspace{2mm}}\hspace{3mm} CH_{3}I\hspace{5mm} + \hspace{5mm} HI$$

As the reaction is reversible, an oxidizing agent such as $HIO_{3}, HNO_{3}, \text{etc.}$ is used.

\section{Chlorination}
$$CH_{4} \hspace{5mm} + \hspace{5mm} Cl_{2} \hspace{3mm} \xrightarrow[\text{Diffused}]{\hspace{2mm} h\nu \hspace{2mm}} \hspace{3mm} CH_{3}Cl \hspace{5mm} + \hspace{5mm} HCl$$
\begin{itemize}
\item $Cl$ is more reactive, hence being less selective.
\item Reactivity ratio $1^\circ : 2^\circ : 3^\circ :: 1:3.8:4.5$
\end{itemize}

\section{Bromination}
$$CH_{4} \hspace{5mm} + \hspace{5mm} Br_{2} \hspace{3mm} \xrightarrow{\hspace{2mm} h\nu \hspace{2mm}} \hspace{3mm} CH_{3}Br \hspace{5mm} + \hspace{5mm} HBr$$

\begin{itemize}
    \item $Br$ being less reactive, hence more selective.
    \item Reactivity ratio $1^\circ : 2^\circ : 3^\circ :: 1:80:1600$
\end{itemize}
\end{document}