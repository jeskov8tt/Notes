\documentclass{article}
\usepackage{graphicx}
\usepackage{amsmath}
\usepackage{amssymb}
\usepackage[italicdiff]{physics}
\usepackage{enumerate}
\usepackage{microtype}
\DisableLigatures{encoding= *, family=*}
\usepackage{titlesec}
\usepackage{xfrac}
\setcounter{secnumdepth}{4}
\usepackage{xcolor}
\usepackage[bookmarks=false]{hyperref}
\usepackage{mathtools}
\hypersetup{
    colorlinks=true,
    linkcolor=[RGB]{59 108 209},
    urlcolor=[RGB]{59 108 209}
}
\urlstyle{same}

\titleformat{\paragraph}
{\normalfont\normalsize\bfseries}{\theparagraph}{1em}{}
\titlespacing*{\paragraph}
{0pt}{3.25ex plus 1ex minus .2ex}{1.5ex plus .2ex}

\title{Electrostatics}
\author{}
\date{}

\begin{document}
\maketitle

\section{Electric Charge}
\begin{itemize}
    \item Charge is characteristic property of fundamental particles due to which it produces and experiences electrical and magnetic effects.
    \item Excess or defficiency of $e^-$ on any body gives concept of charge.
    \item Only $e^-$ are responsible for electrificaton of any body.
    \item SI Unit $1 Amp \cdot 1 Sec=1 Coulomb$,
    
    $1C=3\cdot 10^9 esu, 1emu=10esu, 1esu=1Franklin, 1Faraday=96500C=charge \hspace{1mm} on \hspace{1mm} 1 \hspace{1mm} mol \hspace{1mm} e^-$
\end{itemize}

\section{Properties of Charge}
\begin{enumerate}
    \item Charge is Transferable
    \subitem Induction Method (In case of non-conducting bodies) $$Q_{i}=-Q\left(1-\frac{1}{k}\right) \hspace{3cm} \begin{cases}
        Q_{i}=\text{Induced charge on body} &\\
        Q=\text{Charge on inducing agent} &\\
        k=\text{Dielectric Constant of body being charged}
    \end{cases}$$
\subitem Conduction Method (In case of Conducting bodies)
$$\begin{align}
    Q_{1}=\left(\dfrac{C_{1}}{C_{1}+C_{2}}\right)(q_{1}+q_{2}) &
    Q_{2}=\lef
\end{align}$$
\end{enumerate}

\section{Coulomb's Law}
$$F=\dfrac{1}{4 \pi \epsilon_{0}}\cdot \dfrac{q_{1}q_{2}}{r^2} \hspace{5cm} \dfrac{1}{4 \pi \epsilon_{0}}=9\cdot 10^9 N m^2 C^{-2}$$ 
\end{document}